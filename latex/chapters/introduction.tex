\section{Introduction}
\label{sec:1}
Photovoltaic (PV) systems are a key part of today's clean energy setup, which is crucial in transforming sunlight into electricity through the photovoltaic effect \cite{al2022photovoltaic}. The heart of a PV system is solar panels that generate direct current (DC) electricity when exposed to sunlight \cite{al2025review}. The electricity then will be converted to alternating current (AC) via inverters, making it suitable for residential, commercial, and grid-scale networks \cite{marignetti2023current,rabbani2020solar}. The performance of PV depends on several factors such as solar irradiance, ambient and module temperatures, shading, panel orientation, and system configuration \cite{aslam2022advances,hussin2018performance}. The installation of photovoltaic (PV) systems is typically equipped with advanced monitoring systems that track their operation and surrounding environmental conditions. There are two primary types of monitoring systems: first, the built-in software embedded within the inverter, and the other is external software accessed via the internet, which retrieves data from the inverter every 15 to 45 minutes \cite{iksan2024real}. These systems collect vital information such as grid feed-in, internal power supply, self-consumption, and module temperature. This data provides valuable insights into the system's performance, efficiency, and reliability.

As photovoltaic systems become more complex and widespread, the ability to manage them effectively and enhance performance in real time has become increasingly important. Traditional rule-based or physics-driven models often struggle to accurately detect the dynamic and non-linear behaviors of solar energy systems, especially under varying operational conditions and environmental factors \cite{kumar2024artificial}. To address these limitations, intelligent approaches that utilize both historical and real-time data have emerged, significantly improving overall system performance \cite{kumar2024artificial}. The intelligent approach driven by data, such as machine learning and artificial intelligence, helps to learn a pattern to detect anomalies and make a forecast \cite{salazar2024ai}. These models can process large and complex datasets with high-resolution measurements, such as power output, temperature, and irradiation readings taken at 15-60 minute intervals \cite{kumar2024artificial}. They help identify hidden correlations among data and trends that are not easily detected using traditional analysis methods. By combining both operational and environmental data, the intelligent system can accurately forecast solar power generation, optimize energy usage based on demand and supply conditions, and identify potential faults before they lead to system failures. This approach not only enhances the reliability and efficiency of PV systems but also facilitates smart grid integration and improves energy planning.

While environmental factors such as solar irradiance and temperature have traditionally been primary inputs for forecasting models, high-resolution operational data, including grid feed-in, internal power supply, current power output, module temperature, and self-consumption, can further enhance the intelligence of photovoltaic (PV) systems. This data directly reflects the PV system's behavior and performance under real-world conditions \cite{en17164145}. Integrating operational data into forecasting and diagnostic models allows for a more nuanced understanding of how environmental stimuli and internal dynamics influence PV systems. For example, module temperature and self-consumption patterns can indicate inefficiencies, shading effects, or emerging faults undetectable using weather data alone \cite{daliento2017monitoring}. Furthermore, operational data enables context-aware forecasting, where predictions are tailored to the specific configuration and usage patterns of a particular PV installation. This is exemplified by Zhang et al. (2024), who demonstrate that partitioning data into contextually distinct datasets using Gaussian Mixture Models (GMM) enhances PV power prediction accuracy in LightGBM models \cite{zhang2025integrating}. This improvement originates from the model's ability to leverage feature selection, adapt to varying data distributions through clustering, and optimize power forecasting, which are all essential elements. By training models to recognize patterns and complex anomalies with insights and data, that may lead to system degradation or failure, overall system reliability is enhanced. This approach is particularly beneficial for distributed energy systems, where variability in load profiles and grid interactions can greatly influence generation and consumption dynamics. The increased data production from smart monitoring technologies and IoT-enabled PV systems production motivates the integration of operational data. Utilizing this data not only enhances forecasting accuracy but also supports real-time optimization and proactive maintenance strategies, contributing to the development of truly intelligent solar energy systems.

The objective of this research is to develop a robust and comprehensive framework that enhances the intelligence of photovoltaic (PV) systems by integrating both operational and environmental data. Leveraging a multi-year dataset collected at high-resolution intervals (15-minute and 60-minute), this study addresses critical challenges in solar energy forecasting, system optimization, and fault detection. To ensure the completeness and reliability of the dataset, missing values were systematically filled using a custom data imputation method developed as part of this research. This enriched dataset enables the modeling of real-world PV system behavior with greater accuracy and depth.

Specifically, the research follows the following objectives:
\begin{itemize}
	\item To develop accurate predictive models for solar power generation using a combination of operational and environmental features, improving upon traditional weather-based forecasting approaches.
	\item To design an optimization strategy that utilizes predictive insights to enhance energy efficiency, balance self-consumption, and grid feed-in, and support smarter energy management decisions.
	\item To implement a fault detection mechanism capable of identifying anomalies and early signs of system degradation or malfunction based on patterns in operational data.
	\item To evaluate the effectiveness of machine learning techniques in capturing complex relationships within the dataset and providing interpretable, actionable insights for PV system operators.
\end{itemize}

The key contributions of this study include:
\begin{itemize}
	\item A holistic data-driven framework that integrates diverse data sources for enhanced PV system intelligence.
	\item A comparative analysis of predictive models, highlighting the value of operational data in improving forecast accuracy.
	\item A novel approach to fault detection, leveraging high-frequency operational metrics to identify potential issues before they escalate.
	\item Practical insights and recommendations for real-world deployment of intelligent PV monitoring and management systems.
	\item A custom data completion strategy that ensures the reliability of the dataset used for modeling and analysis.
\end{itemize}
