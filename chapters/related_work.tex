\section{Related work}
\label{sec:2}
\subsection{Overview of Existing Solar Prediction Models}
Solar power prediction is a crucial area of research within renewable energy, involving a range of models from statistical techniques to sophisticated machine learning approaches. Traditional methods like linear regression, ARIMA, and support vector machines (SVM) are commonly used for short-term forecasting. Linear regression, in particular, employs a straightforward equation to relate factors such as sunlight and temperature to solar power output, making it easy to interpret and apply. 

A study by MY Erten et al. (2022) evaluated the performance of various regression models including linear, logistic, Lasso, and elastic regression using a Kaggle dataset. They also explored the use of Principal Component Analysis (PCA) for reducing data dimensions. The results showed that elastic regression combined with PCA achieved the best predictive performance \cite{erten2022solar}. However, the study did not consider other types of models, such as time-series techniques or advanced machine learning algorithms, which limits the comprehensiveness of the analysis. Meanwhile, AK Chaaban et al. (2024) found that linear regression was less effective for predicting solar power compared to other methods. Their results indicated that linear regression produced significant errors, primarily because the relationship between weather variables and solar power is too complex to be accurately modeled with a simple linear approach \cite{chaaban2024comparative}. GF Fan et al. (2022) aimed to enhance the accuracy of photovoltaic (PV) power generation forecasting, particularly in scenarios with high PV integration into the grid. They proposed a hybrid forecasting model that combines ARIMA, backpropagation neural networks (BPNN), and support vector regression (SVR), which offers a more precise and reliable method for predicting PV power output. However, the ARIMA component requires the input data to be stationary, meaning its statistical properties, such as mean, variance, and autocorrelation, remain constant over time. In simpler terms, a stationary series does not exhibit trends or seasonal patterns that change systematically \cite{fan2022photovoltaic}. More recently, deep learning models, particularly Long Short-Term Memory (LSTM) networks and Convolutional Neural Networks (CNNs), have shown superior performance in capturing temporal and spatial patterns in solar irradiance and power output data \cite{alharkan2023solar,ying2023deep,mirjalili2023comparative,guo2024photovoltaic,al2024forecasting}. 

These deep learning models typically rely on environmental inputs such as solar irradiance, temperature, humidity, and cloud cover. While effective to a degree, their accuracy often suffers under rapidly changing weather conditions or in systems with complex operational dynamics. Ensemble methods like Random Forest and XGBoost have also gained popularity for their robustness and ability to handle nonlinear relationships, though they still depend heavily on the quality and diversity of input features.

\subsection{Use of Operational vs. Environmental Data}
Environmental data has traditionally been the cornerstone of solar forecasting models due to its direct influence on photovoltaic performance. However, operational data—such as grid feed-in, internal power supply, current power output, and self-consumption—offers a more granular and system-specific perspective. Studies have begun to explore the integration of these metrics, showing that operational data can significantly enhance model accuracy, especially in scenarios involving partial shading, equipment degradation, or non-standard system configurations.

Despite its potential, operational data remains underutilized in mainstream solar prediction research. Most models treat PV systems as black boxes influenced solely by external conditions, ignoring the rich internal dynamics captured by modern monitoring systems. This gap presents an opportunity to develop more intelligent and context-aware forecasting frameworks.

\subsection{Gaps in Current Research (Prediction, Optimization, Fault Detection)}
Several key gaps persist in the current body of research:
\begin{itemize}
	\item •	Prediction: Many models lack the ability to adapt to system-specific behaviors and seasonal variations. High-frequency data (e.g., 15-minute intervals) is rarely used, limiting the resolution and responsiveness of forecasts.
	\item •	Optimization: Few studies address the operational optimization of PV systems using predictive insights. Most focus on forecasting alone, without translating predictions into actionable strategies for energy management or grid interaction.
	\item •	Fault Detection: Fault detection is often treated as a separate domain, relying on threshold-based alerts or manual inspection. The integration of predictive analytics for early fault identification—using patterns in operational data—is still in its infancy.
\end{itemize}

These gaps highlight the need for holistic models that unify prediction, optimization, and fault detection within a single intelligent framework.

\subsection{Emerging Technologies (AI, IoT, Edge Computing)}
The convergence of artificial intelligence (AI), Internet of Things (IoT), and edge computing is reshaping the landscape of solar energy analytics. AI techniques, particularly deep learning and reinforcement learning, enable systems to learn from historical data and adapt to new conditions. IoT devices facilitate real-time data collection from PV systems, providing a continuous stream of operational and environmental metrics.

Edge computing further enhances this ecosystem by enabling local data processing and decision-making, reducing latency and dependence on cloud infrastructure. This is especially valuable for remote or distributed PV installations where connectivity may be limited.
